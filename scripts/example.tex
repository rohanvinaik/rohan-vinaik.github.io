\documentclass{article}
\usepackage{amsmath}
\usepackage{amssymb}
\usepackage{graphicx}
\usepackage{hyperref}

\title{Example Paper: LaTeX to Website Conversion}
\author{Rohan Vinaik}
\date{January 2025}

\begin{document}

\maketitle

\begin{abstract}
This is an example paper demonstrating the LaTeX to website conversion pipeline. The pipeline uses LaTeXML to convert LaTeX documents to HTML with a custom terminal aesthetic. This abstract showcases the conversion of basic LaTeX elements including text formatting, mathematical equations, figures, tables, and references.
\end{abstract}

\section{Introduction}

This document demonstrates the conversion of LaTeX papers to HTML using the automated pipeline. The system leverages \textbf{LaTeXML}, the same tool used by arXiv for generating accessible HTML versions of research papers.

Key features of the conversion pipeline:
\begin{itemize}
    \item Automated LaTeX to HTML5 conversion
    \item Custom terminal aesthetic styling
    \item MathML support for mathematical content
    \item Figure and table handling
    \item Integration with website infrastructure
\end{itemize}

\section{Mathematical Content}

The pipeline properly handles both inline math like $E = mc^2$ and display equations:

\begin{equation}
\int_{-\infty}^{\infty} e^{-x^2} dx = \sqrt{\pi}
\label{eq:gaussian}
\end{equation}

More complex mathematics is also supported. For example, the Fourier transform:

\begin{equation}
\hat{f}(\omega) = \int_{-\infty}^{\infty} f(t) e^{-i\omega t} dt
\end{equation}

Matrix equations work as well:

\begin{equation}
\mathbf{A}\mathbf{x} = \mathbf{b}
\quad \Rightarrow \quad
\mathbf{x} = \mathbf{A}^{-1}\mathbf{b}
\end{equation}

\section{Theorems and Proofs}

\newtheorem{theorem}{Theorem}
\newtheorem{lemma}{Lemma}
\newtheorem{definition}{Definition}

\begin{definition}[Vector Space]
A vector space $V$ over a field $F$ is a set equipped with two operations (addition and scalar multiplication) satisfying eight axioms.
\end{definition}

\begin{theorem}[Pythagorean Theorem]
In a right triangle, the square of the hypotenuse equals the sum of squares of the other two sides:
\begin{equation}
a^2 + b^2 = c^2
\end{equation}
\end{theorem}

\begin{proof}
This is a placeholder for a proof. In the actual conversion, this would be styled with the terminal aesthetic.
\end{proof}

\section{Tables}

Tables are converted to HTML table elements with proper styling:

\begin{table}[h]
\centering
\begin{tabular}{|l|c|r|}
\hline
\textbf{Method} & \textbf{Accuracy} & \textbf{Speed} \\
\hline
LaTeXML & 95\% & Fast \\
Manual & 100\% & Slow \\
Hybrid & 98\% & Medium \\
\hline
\end{tabular}
\caption{Comparison of conversion methods}
\label{tab:methods}
\end{table}

As shown in Table~\ref{tab:methods}, LaTeXML provides a good balance of accuracy and speed.

\section{Lists and Formatting}

The pipeline handles various list types:

\begin{enumerate}
    \item First item with \textit{italic text}
    \item Second item with \textbf{bold text}
    \item Third item with \texttt{monospace text}
    \begin{enumerate}
        \item Nested item 1
        \item Nested item 2
    \end{enumerate}
\end{enumerate}

\section{Code and Verbatim}

Code blocks are styled with the terminal aesthetic:

\begin{verbatim}
def convert_latex(input_file):
    """Convert LaTeX to HTML."""
    subprocess.run(['latexmlc', input_file])
    return True
\end{verbatim}

\section{Cross-References}

The system preserves cross-references. For example, we can reference Equation~\ref{eq:gaussian} or Table~\ref{tab:methods}.

\section{Conclusion}

This example demonstrates the key features of the LaTeX to website conversion pipeline:

\begin{itemize}
    \item \textbf{Automated conversion}: Single command converts entire papers
    \item \textbf{Semantic preservation}: Mathematical content, structure, and references maintained
    \item \textbf{Custom styling}: Terminal aesthetic applied consistently
    \item \textbf{Integration ready}: Output formatted for direct website integration
\end{itemize}

The pipeline significantly streamlines the process of adding research papers to the website while maintaining high quality and consistent styling.

\section*{Acknowledgments}

This pipeline builds on LaTeXML, developed by the team at NIST and used by arXiv for accessible HTML conversion.

\begin{thebibliography}{9}

\bibitem{latexml}
Miller, B. R.
\textit{LaTeXML: A LaTeX to XML/HTML/MathML Converter}.
NIST, \url{https://dlmf.nist.gov/LaTeXML/}

\bibitem{arxiv}
arXiv.org
\textit{Accessible HTML Papers on arXiv}.
\url{https://info.arxiv.org/about/accessible_HTML.html}

\bibitem{mathjax}
MathJax Consortium
\textit{MathJax: Beautiful and accessible math in all browsers}.
\url{https://www.mathjax.org/}

\end{thebibliography}

\end{document}
