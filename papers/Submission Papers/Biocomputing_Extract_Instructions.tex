\documentclass[11pt]{article}
\usepackage[margin=1in]{geometry}
\usepackage{hyperref}
\usepackage{xcolor}
\usepackage{enumitem}
\usepackage{tcolorbox}

\definecolor{cut}{RGB}{220,20,60}
\definecolor{keep}{RGB}{34,139,34}
\definecolor{compress}{RGB}{255,140,0}

\title{\textbf{Biocomputing Paper → BioSystems:\\Extraction \& Restructuring Guide}}
\author{From 25,000 to 9,000 Words}
\date{\today}

\begin{document}
\maketitle

\section*{Important Note}

\textcolor{cut}{\textbf{DO NOT start this paper yet!}}

Focus on getting COEC submitted first (6 weeks). Then return to this in 3-4 months.

This guide will be here when you're ready.

\section{Executive Summary}

\subsection{The Challenge}
Your biocomputing paper is excellent but too long (25,000 words) and too speculative for immediate publication. It needs to become a focused 8,000-10,000 word research article.

\subsection{The Solution}
Extract the core ``Hyperdimensional Computing in Biological Substrates'' narrative, cutting 60\% while retaining the most novel and defensible contributions.

\subsection{Target Venue: BioSystems}

\textbf{Why BioSystems:}
\begin{itemize}
    \item Explicitly welcomes theoretical frameworks
    \item Focuses on ``self-organizing information systems''
    \item More flexible than ALife on style
    \item Faster review (2-3 months typically)
    \item No length restrictions (but 8-10k words is optimal)
\end{itemize}

\textbf{Scope alignment:}
\begin{quote}
``BioSystems encourages theoretical, computational, and experimental articles that link biology, evolutionary concepts, and the information sciences... biological complexity, theoretical biology, artificial life, computational modeling of complex biological systems.''
\end{quote}

\textbf{Your paper's fit:} ✓✓✓ Perfect match

\section{What to Keep vs. Cut}

\subsection{The Core Narrative to Extract}

\textbf{Focus:} ``Hyperdimensional Computing as a Natural Framework for Biological Information Processing''

\textbf{Central thesis:} Biological systems naturally encode information in high-dimensional spaces (proteins, DNA, gene expression, neural activity), making them ideal substrates for hyperdimensional computing paradigms that offer noise robustness and distributed representation.

\subsection{Detailed Keep/Cut/Compress Map}

\begin{center}
\begin{tabular}{|p{4cm}|p{2cm}|p{1.5cm}|p{5cm}|}
\hline
\textbf{Current Section} & \textbf{Length} & \textbf{Action} & \textbf{Target} \\
\hline
1. Introduction & 2,500 & \textcolor{compress}{COMPRESS} & 1,200 words - Cut multi-substrate emphasis, focus on HDC in biology \\
\hline
2.1 Proteins \& ncAAs & 2,000 & \textcolor{keep}{KEEP} & 1,500 words - Core contribution \\
\hline
2.2 DNA & 1,500 & \textcolor{compress}{COMPRESS} & 800 words - Less detail on implementation \\
\hline
2.3 RNA & 800 & \textcolor{compress}{COMPRESS} & 400 words - Brief mention \\
\hline
2.4 Viruses & 1,000 & \textcolor{cut}{CUT} & 0 - Too speculative, save for future \\
\hline
2.5 Organelles & 1,200 & \textcolor{cut}{CUT} & 0 - Peripheral to HDC focus \\
\hline
2.6 Neurons & 1,500 & \textcolor{keep}{KEEP} & 1,000 - Natural HDC substrate \\
\hline
2.7 Hormones & 800 & \textcolor{cut}{CUT} & 0 - Not central to HDC \\
\hline
3. HDC Theory & 4,000 & \textcolor{keep}{KEEP} & 3,000 - Your strongest section \\
\hline
4. Algorithms & 3,500 & \textcolor{compress}{COMPRESS} & 1,500 - Keep key algorithms, cut exhaustive coverage \\
\hline
5. Implementation & 2,500 & \textcolor{cut}{CUT} & 100 - Brief ``future work'' mention \\
\hline
6. Feasibility & 2,000 & \textcolor{cut}{CUT} & 200 - Move to ``challenges'' subsection \\
\hline
7. Applications & 2,200 & \textcolor{compress}{COMPRESS} & 800 - Focus on Kimaiya only \\
\hline
8. Future & 500 & \textcolor{cut}{CUT} & 0 - Fold into conclusion \\
\hline
9. Conclusion & 500 & \textcolor{keep}{KEEP} & 600 - Expand slightly \\
\hline
\textbf{TOTAL} & \textbf{25,000} &  & \textbf{9,000} \\
\hline
\end{tabular}
\end{center}

\section{New Paper Structure}

\subsection{Title}

\textbf{Suggested:}\\
\textcolor{keep}{Hyperdimensional Computing in Biological Substrates: A Framework for Understanding Natural Information Processing}

\textbf{Alternative:}\\
\textcolor{keep}{Biological Hyperdimensional Computing: From Molecular Encoding to Neural Dynamics}

\subsection{Abstract (250 words)}

\begin{tcolorbox}[colback=keep!10,colframe=keep,title=New Abstract Template]
Biological systems encode and process information across multiple scales and substrates, from protein conformations to neural population codes. We present a theoretical framework demonstrating how this natural encoding maps onto hyperdimensional computing (HDC) paradigms, where information is distributed across very high-dimensional vector spaces. Unlike traditional low-dimensional digital abstractions, HDC naturally captures the rich computational capacity inherent in biological systems: proteins encode information through amino acid sequences ($\sim$4.3 bits/position), conformational dynamics (hundreds of dimensions), and post-translational modifications (exponential state spaces); DNA sequences span $4^L$ discrete states with thermodynamic structure; gene expression profiles occupy tens-of-thousands-dimensional spaces; and neural populations represent information through millions to billions of activity dimensions.

We formalize how biological substrates implement core HDC operations—binding (composition), bundling (superposition), and permutation (sequencing)—through physical mechanisms including allosteric regulation, strand displacement, and synaptic integration. The framework establishes connections between HDC's mathematical properties (noise tolerance, associative memory, compositionality) and biological computation's observed characteristics (robustness, distributed processing, graceful degradation).

We demonstrate the framework's utility through analysis of protein design with synthetic amino acid expansion, showing how increasing encoding alphabet from 20 to 30 amino acids expands computational capacity while maintaining biological functionality. Case studies in stem cell differentiation (Kimaiya platform) and neuromorphic computing illustrate how HDC principles guide both analysis and engineering of biological computation. This framework provides both analytical tools for understanding natural biological information processing and design principles for synthetic biological computing systems.
\end{tcolorbox}

\subsection{Section-by-Section Structure}

\subsubsection{1. Introduction (1,200 words)}

\textbf{1.1 Biological Computation Beyond Digital Logic (300 words)}
\begin{itemize}
    \item Living systems perform computation without symbolic logic
    \item Traditional CS frameworks inadequately capture biological richness
    \item Need for frameworks matching biological reality
\end{itemize}

\textbf{1.2 Hyperdimensional Computing as Natural Match (400 words)}
\begin{itemize}
    \item HDC: distributed representation in very high dimensions
    \item Key properties: noise tolerance, similarity operations, compositionality
    \item Biological systems naturally high-dimensional
    \item Hypothesis: biological encoding evolved for HDC-like computation
\end{itemize}

\textbf{1.3 Framework Overview and Contributions (500 words)}
\begin{itemize}
    \item Multi-level encoding in proteins, DNA, neurons
    \item Mapping to HDC operations
    \item Design principles for synthetic systems
    \item Case studies demonstrating utility
\end{itemize}

\subsubsection{2. Biological Substrates as HDC Implementers (3,700 words)}

\textbf{2.1 Proteins and Synthetic Amino Acid Expansion (1,500 words)}
\begin{itemize}
    \item Multi-level encoding: sequence, structure, PTMs
    \item Conformational hyperdimensionality
    \item Synthetic ncAAs expanding encoding capacity
    \item Mathematical formalization of state spaces
    \item \textcolor{keep}{KEEP this section mostly intact—it's your strongest}
\end{itemize}

\textbf{2.2 Nucleic Acids (800 words)}
\begin{itemize}
    \item DNA sequence space and encoding capacity
    \item RNA structure and function
    \item Strand displacement as HDC operation
    \item Thermodynamic embeddings
    \item \textcolor{compress}{COMPRESS from 2,300 to 800 words}
\end{itemize}

\textbf{2.3 Neural Population Codes (1,000 words)}
\begin{itemize}
    \item Neural activity as high-dimensional vectors
    \item Synaptic integration as binding/bundling
    \item Noise tolerance in neural codes
    \item Connection to predictive processing
    \item \textcolor{keep}{KEEP but compress from 1,500}
\end{itemize}

\textbf{2.4 Comparative Analysis (400 words)}
\begin{itemize}
    \item Dimensionality comparison across substrates
    \item Noise characteristics
    \item Operation speeds
    \item Table summarizing properties
    \item \textcolor{keep}{ADD this new section for coherence}
\end{itemize}

\subsubsection{3. Hyperdimensional Operations in Biological Systems (3,000 words)}

\textbf{3.1 Theoretical Foundations (800 words)}
\begin{itemize}
    \item HDC mathematical framework recap
    \item Binding, bundling, permutation operations
    \item Noise tolerance properties
    \item \textcolor{compress}{COMPRESS your Section 3.1—less general, more focused}
\end{itemize}

\textbf{3.2 Protein-Based HDC (1,000 words)}
\begin{itemize}
    \item Conformational vectors
    \item PTMs as operations
    \item Complex formation as composition
    \item \textcolor{keep}{KEEP Section 3.2 mostly intact}
\end{itemize}

\textbf{3.3 Nucleic Acid HDC (600 words)}
\begin{itemize}
    \item Sequence embeddings
    \item Structural encodings
    \item Concentration-based vectors
    \item \textcolor{compress}{COMPRESS from Section 3.3}
\end{itemize}

\textbf{3.4 Neural HDC (600 words)}
\begin{itemize}
    \item Population coding
    \item Synaptic operations
    \item Temporal binding
    \item \textcolor{keep}{Light edit of your material}
\end{itemize}

\subsubsection{4. Computational Framework (1,500 words)}

\textbf{4.1 Multi-Scale Modeling (500 words)}
\begin{itemize}
    \item ODEs for molecular dynamics
    \item PDEs for spatial structure
    \item Stochastic processes
    \item \textcolor{compress}{Heavy compression from Section 4}
\end{itemize}

\textbf{4.2 Key Algorithms (600 words)}
\begin{itemize}
    \item Large-scale equation solving
    \item Network flow optimization
    \item Topological data analysis
    \item \textcolor{compress}{Pick 3 most relevant, brief descriptions}
\end{itemize}

\textbf{4.3 Machine Learning Integration (400 words)}
\begin{itemize}
    \item Predictive modeling
    \item RL for control
    \item Pruned networks for interpretation
    \item \textcolor{compress}{Very brief, essential points only}
\end{itemize}

\subsubsection{5. Case Study: Synthetic Biology Application (800 words)}

\textbf{Focus exclusively on Kimaiya platform:}
\begin{itemize}
    \item HDC principles in stem cell differentiation
    \item High-dimensional epigenetic state space
    \item Constraint-guided trajectory optimization
    \item Results: 95\% efficiency, reduced timeframes
    \item Demonstrates framework utility
\end{itemize}

\textcolor{cut}{CUT all other applications—too scattered}

\subsubsection{6. Discussion (800 words)}

\textbf{6.1 Implications for Understanding Biological Computation (400 words)}
\begin{itemize}
    \item Why HDC perspective matters
    \item New insights into biological information processing
    \item Connections to active inference, FEP
\end{itemize}

\textbf{6.2 Design Principles for Synthetic Systems (200 words)}
\begin{itemize}
    \item HDC-guided engineering
    \item Substrate selection criteria
    \item Noise management strategies
\end{itemize}

\textbf{6.3 Challenges and Limitations (200 words)}
\begin{itemize}
    \item Measurement difficulties
    \item Readout latency
    \item Integration complexity
    \item \textcolor{compress}{Condense from feasibility section}
\end{itemize}

\subsubsection{7. Conclusion (600 words)}

\begin{itemize}
    \item Framework recap
    \item Key insights
    \item Future research directions (brief, 150 words max)
    \item Broader implications
\end{itemize}

\section{Major Prose Changes Needed}

\subsection{Introduction Rewrite}

\textbf{Current opening:}
\begin{quote}
``Biological computation represents a paradigm shift...''
\end{quote}

\textbf{New opening:}
\begin{tcolorbox}[colback=keep!10,colframe=keep]
From molecular self-assembly to neural computation, living systems process information through mechanisms that appear fundamentally different from traditional digital computers. Rather than manipulating discrete symbols through logic gates, biological computation emerges from the physical dynamics of high-dimensional molecular systems—protein conformations, genetic regulatory networks, neural population codes—that naturally implement distributed, analog information processing. Understanding and engineering these systems requires computational frameworks that match their intrinsic properties.

Hyperdimensional computing (HDC) offers a promising lens for understanding biological information processing. In HDC, information is represented as points in very high-dimensional vector spaces (typically thousands to millions of dimensions), with similarity encoded through geometric relationships. Operations on these representations—composition through binding, superposition through bundling, sequencing through permutation—naturally emerge from the mathematical structure of high-dimensional spaces. Critically, HDC systems exhibit noise tolerance, graceful degradation, and associative memory properties that mirror biological computation's observed characteristics.

Here we demonstrate that biological substrates—from proteins to neural networks—naturally implement HDC principles through their physical properties...
\end{tcolorbox}

\subsection{Section 2 Compression Strategy}

For each substrate section, follow this template:

\begin{tcolorbox}[colback=compress!10,colframe=compress,title=Substrate Section Template (300-500 words each)]
\textbf{Paragraph 1:} Physical structure and natural encoding properties

\textbf{Paragraph 2:} Dimensionality analysis—how many effective dimensions?

\textbf{Paragraph 3:} How this substrate implements HDC operations

\textbf{Paragraph 4:} Noise characteristics and robustness properties

\textbf{Optional Paragraph 5:} Example or case study (only if compelling)
\end{tcolorbox}

\subsection{Section 3 Focus}

Currently Section 3 is very thorough but somewhat repetitive. 

\textcolor{compress}{\textbf{Compression strategy:}}
\begin{itemize}
    \item Combine Sections 3.1 and 3.2 into unified theoretical foundation (800 words)
    \item One focused subsection per substrate type (600-1000 words each)
    \item Remove redundant explanations across subsections
    \item Use consistent notation throughout
\end{itemize}

\subsection{Section 4 Ruthless Pruning}

\textcolor{cut}{\textbf{Cut entirely:}}
\begin{itemize}
    \item Detailed algorithm descriptions (keep 1-2 sentences each)
    \item Complexity analysis (mention briefly)
    \item Implementation details
    \item Extensive citations to computer science papers
\end{itemize}

\textcolor{keep}{\textbf{Keep:}}
\begin{itemize}
    \item High-level overview of computational tools
    \item Why these tools matter for biological HDC
    \item Connection to framework
\end{itemize}

\textbf{Target:} Reduce from 3,500 to 1,500 words

\section{What Makes This Work for BioSystems}

\subsection{Alignment with Journal Scope}

BioSystems wants papers that:
\begin{itemize}
    \item ✓ Link biology and information sciences (you do this)
    \item ✓ Address self-organizing information systems (HDC is self-organizing)
    \item ✓ Provide theoretical frameworks (your core contribution)
    \item ✓ Have biological relevance (grounded in real systems)
    \item ✓ Offer computational insights (HDC lens is novel)
\end{itemize}

\subsection{Competitive Advantages}

Your paper offers:
\begin{enumerate}
    \item \textbf{Novel theoretical lens:} HDC hasn't been systematically applied to biological systems at this level
    \item \textbf{Cross-scale unification:} Proteins to neural networks under one framework
    \item \textbf{Practical demonstrations:} Kimaiya shows framework utility
    \item \textbf{Rigorous formalism:} Mathematical precision with biological grounding
\end{enumerate}

\section{Timeline for This Paper}

\textbf{Don't start until COEC is submitted!}

Once COEC is submitted (Week 6):
\begin{itemize}
    \item Week 7-8: Take a break, refresh
    \item Week 9-12: Extract and restructure (following this guide)
    \item Week 13-14: Polish and format
    \item Week 15: Submit to BioSystems
\end{itemize}

\textbf{Total timeline:} 3-4 months after starting COEC

\section{Extraction Checklist}

\subsection{Before Starting}
\begin{itemize}[label=$\square$]
    \item COEC paper submitted to ALife
    \item Read 3-4 recent BioSystems papers
    \item Have current biocomputing HTML open
    \item Block out time for focused work
\end{itemize}

\subsection{Major Tasks}
\begin{itemize}[label=$\square$]
    \item Create new document in Overleaf
    \item Write new introduction (1,200 words)
    \item Extract and compress Section 2 substrates
    \item Refocus Section 3 on HDC operations
    \item Condense Section 4 algorithms
    \item Expand Kimaiya case study
    \item Write new discussion
    \item Polish conclusion
    \item Format for BioSystems
\end{itemize}

\subsection{Final Checks}
\begin{itemize}[label=$\square$]
    \item Length: 8,000-10,000 words
    \item Focused narrative (HDC in biology)
    \item No viral/organelle sections
    \item Minimal implementation details
    \item One strong case study (Kimaiya)
    \item References to BioSystems papers
    \item Abstract emphasizes novelty
\end{itemize}

\section{Why This Extraction Works}

\subsection{From Speculative to Defensible}

\textbf{Current problem:} Paper tries to do too much
\begin{itemize}
    \item Multiple substrates with varying feasibility
    \item Implementation details that don't exist yet
    \item Applications across too many domains
    \item Feels like 3-4 papers merged together
\end{itemize}

\textbf{Extracted solution:} Focused, coherent narrative
\begin{itemize}
    \item Single clear thesis: biological systems are natural HDC implementers
    \item Core substrates only (proteins, DNA/RNA, neurons)
    \item Theoretical framework with one strong demonstration
    \item Defensible at every point
\end{itemize}

\subsection{What You Can Still Publish Later}

The material you're cutting isn't wasted—it forms the basis for 2-3 future papers:

\textbf{Paper 3:} ``Multi-Substrate Integration for Biocomputing''
\begin{itemize}
    \item Focus on viruses, organelles, hormones
    \item Integration strategies
    \item Target: \textit{ACS Synthetic Biology} or \textit{Synthetic Biology}
\end{itemize}

\textbf{Paper 4:} ``Implementation Framework for Biological HDC''
\begin{itemize}
    \item Focus on Section 5-6 (implementation, feasibility)
    \item Detailed engineering considerations
    \item Target: \textit{Nature Communications} or \textit{eLife}
\end{itemize}

\textbf{Paper 5:} ``Applications of Biological HDC''
\begin{itemize}
    \item Expand all application sections
    \item Multiple case studies
    \item Target: \textit{Trends in Biotechnology} (review)
\end{itemize}

\section{BioSystems Submission Requirements}

\subsection{Format}
\begin{itemize}
    \item LaTeX or Word (LaTeX preferred)
    \item No strict template (more flexible than ALife!)
    \item Standard academic formatting
    \item References: author-date or numbered (both acceptable)
\end{itemize}

\subsection{Submission}
\begin{itemize}
    \item Online via Elsevier Editorial System
    \item Cover letter recommended but not required
    \item Suggest reviewers (optional, helpful)
\end{itemize}

\subsection{Review Process}
\begin{itemize}
    \item Initial decision: 1-2 weeks
    \item Full review: 2-3 months (faster than ALife)
    \item Acceptance rate: ~35-40\%
    \item Revisions common and expected
\end{itemize}

\section{Sample Cover Letter}

\begin{tcolorbox}[colback=blue!10,colframe=blue]
Dear Editor,

I submit for consideration in \textit{BioSystems} the manuscript ``Hyperdimensional Computing in Biological Substrates: A Framework for Understanding Natural Information Processing.''

This work presents a novel theoretical framework demonstrating how biological systems—from proteins to neural networks—naturally implement hyperdimensional computing principles through their physical properties. By formalizing the mapping between biological encoding mechanisms and HDC operations, this framework provides both analytical tools for understanding biological information processing and design principles for synthetic biological computing systems.

Key contributions include:
\begin{itemize}
    \item Mathematical formalization of biological substrates as HDC implementers
    \item Analysis of how protein conformations, nucleic acid structures, and neural codes encode information in high-dimensional spaces
    \item Demonstration of framework utility through stem cell differentiation case study (95\% efficiency achieved)
    \item Design principles for engineering biological computing systems
\end{itemize}

This work addresses \textit{BioSystems}' core themes of self-organizing information systems and theoretical biology, offering a unifying computational perspective on biological information processing across scales.

Thank you for your consideration.

Sincerely,\\
Rohan Vinaik
\end{tcolorbox}

\section{Success Metrics}

\subsection{How You'll Know This Paper is Ready}

\textbf{Content metrics:}
\begin{itemize}
    \item 8,000-10,000 words total
    \item Single clear thesis throughout
    \item 3 core substrates, well integrated
    \item 1 strong case study
    \item No hand-waving or speculation
\end{itemize}

\textbf{Quality checks:}
\begin{itemize}
    \item Can explain paper in 2 minutes to non-expert
    \item Every section advances main argument
    \item No orphaned content
    \item Figures clearly support text
    \item References support every claim
\end{itemize}

\textbf{Submission readiness:}
\begin{itemize}
    \item Formatted cleanly
    \item No typos or formatting errors
    \item All citations complete
    \item Cover letter written
    \item You're proud of it!
\end{itemize}

\section{Final Encouragement}

This paper contains genuinely novel ideas about biological computation. The main challenge is focus—transforming an encyclopedic survey into a sharp theoretical contribution.

The extraction process will be challenging but rewarding:
\begin{itemize}
    \item You'll clarify your own thinking
    \item The paper will be stronger and more defensible
    \item You'll have clearer ideas for follow-up papers
    \item BioSystems acceptance is realistic (55-65\% chance)
\end{itemize}

\textbf{Most importantly:} You're not abandoning the cut material. It forms the foundation for 2-3 more papers. You're building a publication pipeline, not losing work.

\vspace{2em}
\hrule
\vspace{1em}

\textit{Remember: Focus on COEC first! Return to this in 3-4 months when you're ready for the next challenge.}

\end{document}
