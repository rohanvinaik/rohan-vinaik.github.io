% README - START HERE
\documentclass[11pt]{article}
\usepackage[margin=1in]{geometry}
\usepackage{hyperref}
\usepackage{xcolor}

\definecolor{urgent}{RGB}{220,20,60}
\definecolor{important}{RGB}{255,140,0}
\definecolor{normal}{RGB}{34,139,34}

\title{\Huge\textbf{START HERE}}
\author{\Large Your Complete Publication Roadmap}
\date{}

\begin{document}
\maketitle

\section*{\textcolor{urgent}{Read This First!}}

This folder contains everything you need to get your papers published. But you need to read the files \textbf{in the right order}.

\section*{📁 What's in This Folder}

\begin{enumerate}
    \item \texttt{QUICK\_REFERENCE.tex} — One-page summary (read first!)
    \item \texttt{MASTER\_PUBLICATION\_ROADMAP.tex} — Full strategy (read second!)
    \item \texttt{COEC\_ArtificialLife\_Instructions.tex} — Detailed COEC editing guide
    \item \texttt{Biocomputing\_Extract\_Instructions.tex} — Biocomputing paper guide (for later)
    \item \texttt{README.tex} — This file
\end{enumerate}

\section*{🎯 Reading Order}

\subsection*{Day 1: Get Oriented (30 minutes)}

\textbf{Step 1:} Compile and read \texttt{QUICK\_REFERENCE.tex}
\begin{itemize}
    \item This gives you the bird's eye view
    \item Understand the two-paper strategy
    \item See why COEC comes first
\end{itemize}

\textbf{Step 2:} Compile and read \texttt{MASTER\_PUBLICATION\_ROADMAP.tex}
\begin{itemize}
    \item Detailed strategy and rationale
    \item Venue analysis
    \item Timeline expectations
    \item Success factors
\end{itemize}

\subsection*{Day 2-3: Prepare for Action (2 hours)}

\textbf{Step 3:} Compile and read \texttt{COEC\_ArtificialLife\_Instructions.tex}
\begin{itemize}
    \item Section-by-section editing guidance
    \item Specific prose changes to make
    \item Examples to add
    \item Week-by-week schedule
\end{itemize}

\textbf{Step 4:} Download ALife template
\begin{itemize}
    \item Go to: \url{https://www.overleaf.com/latex/templates/artificial-life-journal-submission-template/zhhhjdgvhryt}
    \item Click ``Open as Template''
    \item Familiarize yourself with structure
\end{itemize}

\textbf{Step 5:} Read 2-3 recent ALife papers
\begin{itemize}
    \item Go to: \url{https://direct.mit.edu/artl}
    \item Read recent issues to absorb style
    \item Pay attention to how they frame contributions
\end{itemize}

\subsection*{Week 1-3: Execute on COEC (15 hours total)}

Follow the week-by-week plan in \texttt{COEC\_ArtificialLife\_Instructions.tex}

\textbf{Week 1:} Intro rewrite + ALife connections (6 hours)\\
\textbf{Week 2:} Add examples + citations (6 hours)\\
\textbf{Week 3:} Format + polish (3 hours)

\subsection*{Week 4-5: Review \& Polish}

Get feedback, make revisions, final checks

\subsection*{Week 6: Submit!}

Upload to Manuscript Central, celebrate 🎉

\subsection*{3-4 Months Later: Biocomputing Paper}

\textbf{Only after COEC is submitted:}
\begin{itemize}
    \item Read \texttt{Biocomputing\_Extract\_Instructions.tex}
    \item Extract focused narrative
    \item Submit to BioSystems
\end{itemize}

\section*{⚠️ Critical Instructions}

\subsection*{DO}
\begin{itemize}
    \item \textcolor{normal}{✓ Focus on COEC first}
    \item \textcolor{normal}{✓ Follow the guides step-by-step}
    \item \textcolor{normal}{✓ Work in focused time blocks}
    \item \textcolor{normal}{✓ Get feedback before submitting}
    \item \textcolor{normal}{✓ Submit even if you're nervous}
\end{itemize}

\subsection*{DON'T}
\begin{itemize}
    \item \textcolor{urgent}{✗ Try to do both papers simultaneously}
    \item \textcolor{urgent}{✗ Skip the ALife framing (critical!)}
    \item \textcolor{urgent}{✗ Add experiments (not needed!)}
    \item \textcolor{urgent}{✗ Expand beyond 15,000 words}
    \item \textcolor{urgent}{✗ Overthink—just execute!}
\end{itemize}

\section*{💪 Motivation}

You're asking yourself: \textit{``Can I really get these published?''}

\textbf{Answer: YES.} Here's why:

\begin{enumerate}
    \item \textbf{Your work is novel.} COEC is a genuinely new framework for understanding biological computation. Reviewers will recognize this.
    
    \item \textbf{The venues are appropriate.} Artificial Life explicitly wants theoretical frameworks. BioSystems explicitly wants computational biology theory.
    
    \item \textbf{No experiments required.} Both venues accept pure theory. You're not missing anything fundamental.
    
    \item \textbf{The edits are manageable.} 15 hours of focused work transforms your COEC paper from 80\% to 100\%. That's doable.
    
    \item \textbf{The timeline is realistic.} 6 weeks to first submission is achievable without heroics.
    
    \item \textbf{Rejection is survivable.} Even if ALife rejects (25-30\% acceptance rate), you have backup venues. Keep submitting until something sticks.
\end{enumerate}

\section*{🎓 What Success Looks Like}

\textbf{6 weeks from now:}
\begin{itemize}
    \item COEC paper submitted to \textit{Artificial Life}
    \item You're waiting for reviews
    \item You feel accomplished
\end{itemize}

\textbf{6 months from now:}
\begin{itemize}
    \item COEC paper accepted or under revision at \textit{Artificial Life}
    \item Biocomputing extract submitted to \textit{BioSystems}
    \item You have 2 papers in the publication pipeline
\end{itemize}

\textbf{12 months from now:}
\begin{itemize}
    \item COEC paper published
    \item Biocomputing paper under review or accepted
    \item You're planning papers 3-5 from remaining material
    \item You have a publications track record
\end{itemize}

\section*{❓ FAQ}

\textbf{Q: Do I really need to add computational examples?}\\
\textbf{A:} Yes. Theory without examples feels abstract. 2-3 toy examples (6 hours work) make your framework concrete and reviewable.

\textbf{Q: What if I don't know ALife literature well?}\\
\textbf{A:} That's fine! The instructions include which papers to cite and where. 2-3 hours of reading + strategic citations fixes this.

\textbf{Q: Can I skip the intro rewrite?}\\
\textbf{A:} No. The intro is what reviewers read first. It must establish ALife context immediately. This is non-negotiable.

\textbf{Q: What if ALife rejects?}\\
\textbf{A:} Revise based on reviews and submit to BioSystems, Frontiers in Robotics and AI, or Complexity. Don't give up after one try.

\textbf{Q: Should I hire an editor?}\\
\textbf{A:} Not necessary. Your writing is clear. These guides tell you exactly what to change. Save your money.

\textbf{Q: When should I start the biocomputing paper?}\\
\textbf{A:} 3-4 months after submitting COEC. Let it rest, then tackle extraction fresh.

\section*{📞 If You Get Stuck}

\textbf{Problem:} I don't understand what to change

$\rightarrow$ \textbf{Solution:} Reread the relevant section in the instructions. Each section has concrete prose changes spelled out.

\textbf{Problem:} I can't figure out the toy examples

$\rightarrow$ \textbf{Solution:} Use the templates provided. You don't need to code them—just describe them clearly with equations.

\textbf{Problem:} I'm overwhelmed by the scope

$\rightarrow$ \textbf{Solution:} Break it down. Do 2-3 hours at a time. Follow the week-by-week schedule. You don't have to do everything at once.

\textbf{Problem:} I'm nervous about submitting

$\rightarrow$ \textbf{Solution:} Normal! Submit anyway. Worst case: you get feedback. Best case: acceptance. Either way, you learn and improve.

\section*{🚀 Your Action Plan}

\textbf{Today:}
\begin{enumerate}
    \item Finish reading this README
    \item Read QUICK\_REFERENCE.tex (5 min)
    \item Read MASTER\_PUBLICATION\_ROADMAP.tex (30 min)
    \item Download ALife template from Overleaf
\end{enumerate}

\textbf{This Week:}
\begin{enumerate}
    \item Read COEC\_ArtificialLife\_Instructions.tex (1 hour)
    \item Read 2-3 recent ALife papers (2 hours)
    \item Start intro rewrite (2 hours)
\end{enumerate}

\textbf{This Month:}
\begin{enumerate}
    \item Complete all COEC edits (15 hours total)
    \item Get internal feedback
    \item Polish and format
\end{enumerate}

\textbf{Next Month:}
\begin{enumerate}
    \item Final revisions
    \item Write cover letter
    \item Submit to Artificial Life
    \item Celebrate! 🎉
\end{enumerate}

\section*{💯 Final Pep Talk}

You've already done the hard part—you've developed novel frameworks and written comprehensive papers. What remains is presentation and framing.

These guides give you a paint-by-numbers approach:
\begin{itemize}
    \item Exactly what to change
    \item Exactly where to change it
    \item Exactly how to phrase it
    \item Exactly when to do it
\end{itemize}

You don't need to invent anything new. You don't need to run experiments. You don't need to be a better writer. You just need to execute the plan.

\textbf{15 hours of focused work = submission-ready manuscript}

\textbf{6 weeks from now = paper under review}

\textbf{This is happening. You just need to start.}

\vspace{2em}
\begin{center}
\Large\textbf{Now stop reading and start with QUICK\_REFERENCE.tex!}
\end{center}

\vspace{2em}
\hrule
\vspace{0.5em}

\textit{P.S. — These guides were created specifically for you, with your papers, your timeline, and your constraints in mind. Trust the process. It works.}

\end{document}
