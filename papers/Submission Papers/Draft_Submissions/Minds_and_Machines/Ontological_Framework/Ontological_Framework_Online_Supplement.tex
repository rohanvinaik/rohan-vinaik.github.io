\documentclass[12pt]{article}
\usepackage[utf8]{inputenc}
\usepackage[margin=1in]{geometry}
\usepackage{setspace}
\usepackage{amsmath}
\usepackage{amssymb}
\usepackage{amsthm}
\usepackage{natbib}
\doublespacing
\bibliographystyle{apalike}

\title{Online Supplementary Materials:\\
An Ontological Framework for Meaning, Knowledge, and Intelligence}
\author{Rohan Vinaik}
\date{}

\begin{document}
\maketitle

\section*{Note to Readers}
This supplement contains extended discussions, detailed examples, comprehensive literature reviews, and additional philosophical objections that were moved from the main paper to meet venue length requirements. All content here supports but is not essential to the core arguments presented in the main paper.

\section{Extended Literature Review}
% Extended related work and comprehensive survey of relevant literature

\section{Additional Philosophical Background}
% Detailed background on ontology, meaning theory, agent-based cognition

\section{Extended Examples and Case Studies}

\subsection{Illustrative Vignettes}

\textit{These vignettes are strictly illustrative applications of the framework; they do not drive the argument or carry evidentiary weight beyond exemplification.}

\textbf{Mode I — Sacred Silence (Soviet Anti-Aesthetic).} Soviet material culture functions as art through ostensible lack of art: meaning appears via disciplined negation rather than ornament, yielding "meaningful absence" with coherent suppression and sacred orientation \citep{groys1992total}. Semantic Density: high via negation; Interaction: coherent suppression; Intentionality: sacred (ideological devotion), producing the felt "haunted chapel."

\textbf{Mode II — Emergent Chaos (The Big Lebowski).} Definition arises from collisions among agents without central narrative authority \citep{coen1998lebowski}. Semantic Density: negative (via anti-narrative structure); Interaction: emergent through character collisions; Intentionality: profane (mundane transgression), resulting in "distributed consciousness without center."

\textbf{Mode III — Positive Construction (The Simpsons, "Do It For Her").} A classical arc with clear motivation and "earned sentiment" aligns narrative vectors; coherent agent interaction and constructive intentionality generate stable meaning \citep{groening1993simpsons}.

\textbf{Mode IV — Generative Constraint (Terminator 2).} Meaning grows through participation under scaffolded constraints and devotional intentionality—"machines learning humanity through narrative participation" \citep{cameron1991terminator}. Transformative Semantic Density, Developmental Interaction, and Devotional Intentionality produce positive $\Delta G$ for characters and audiences.

\textbf{Mode V — Semantic Vampirism (Algorithmically Generated Content).} Consider AI-generated "content farms" optimized for search engines: articles that mimic informational text structure (headings, listicles, citations) while providing minimal novel information \citep{goldstein2023generative}. Surface similarity to functional content coexists with mechanical reproduction of form and absence of grounded purpose—an anti-life vector (extractive intent) that actively drains semantic space by displacing genuine information sources. Semantic Density: vampiric (negative contribution to ecosystem); Interaction: mechanical (template-filling); Intentionality: anti-life (extractive), yielding the "synthetic morgue" effect.

\section{Detailed Proofs and Formal Derivations}
% Mathematical proofs and formal derivations

\section{Extended Discussion of Contemporary AI Systems}
% Additional analysis of GPT-4, Claude, Gemini, and other LLMs

\section{Philosophical Objections: Extended Responses}
% Detailed responses to objections with comprehensive argumentation

\section{Additional Implications and Future Directions}
% Extended discussion of implications beyond those in main paper

\section{Implementation and Measurement Details}
% Technical details of operationalization and measurement procedures

\section{Limitations and Scope Conditions}

\subsection{Conceptual Limits}

The framework supplies an \textit{analytic ontology} for comparing systems; it does not claim a final metaphysical account of meaning. Treating constraints, topology, and intentional vectors as sufficient generators is a useful idealization—actual meaning-making likely involves additional factors (e.g., embodied affect, unconscious processes, material substrates).

The sacred–profane–indifferent–anti-life intentionality spectrum is a \textit{descriptive axis}, not a universal moral taxonomy. What counts as "life-affirming" varies across cultures and contexts. Cross-cultural calibration required.

\subsection{Measurement and Identifiability}

Metrics such as $M$, $C_T$, $A$, and $V$ are \textit{proxies} rather than direct measurements of abstract constructs. Different operationalizations may yield different rankings. Unitization introduces coder subjectivity—what counts as a semantic agent depends on grain and domain knowledge.

Different $\langle K, T, I \rangle$ configurations may yield observationally similar signatures (\textit{equifinality}). Mode inference from metrics is probabilistic, not deterministic. Temporal grain matters: short observation windows can misclassify transient turbulence as stable regimes.

Inter-rater reliability for intentional classifications will be lower than for structural features. This is inherent to interpretive work but limits objectivity claims. Triangulation across multiple methods recommended.

\subsection{Domain Constraints}

The framework applies best when:
\begin{itemize}
\item Agent interactions are meaningful and identifiable (not pure noise)
\item Constraints are observable or inferrable
\item Intentional traces are available (authorial statements, design documents, user interviews)
\end{itemize}

For artifacts with minimal relational structure (e.g., raw sensor data, random noise), the agent graph may be too sparse for meaningful analysis. The framework is designed for \textit{sense-making systems}, not unstructured data.

\subsection{Cultural Variability}

Meanings, constraints, and intentionality are culturally embedded. What constitutes "productive constraint" in one tradition may be "empty" in another. High alignment $A$ can reflect coherent but harmful ends (e.g., propaganda). Plural vocabularies and cross-cultural validation essential for generalizing beyond WEIRD (Western, Educated, Industrialized, Rich, Democratic) contexts \citep{henrich2010weirdest}.

The framework's mechanisms ($K$, $T$, $I$) are proposed as universal, but their interpretations and valuations vary. Future work should engage Indigenous epistemologies, non-Western semiotics, and decolonial perspectives to refine and test cultural robustness.

\subsection{Goodharting and Adversarial Behavior}

As per Goodhart's Law \citep{goodhart1984problems}, once metrics become targets, they cease to be good measures. Systems can optimize to look coherent (high $C_T$, low $V$) without genuine gains in $M$. Adversarial mimicry can evade $V$-gates.

\textbf{Countermeasures:}
\begin{itemize}
\item Rotate metric families to prevent static optimization targets
\item Contrastive training against known Mode V exemplars
\item Human-in-the-loop validation for high-stakes decisions
\item Transparency: make metrics and their limitations publicly known
\end{itemize}

\subsection{Computational Tractability}

Computing $C_T$ for large graphs is $O(n^2)$ or worse; relational compression for $M$ may require expensive inference. Approximations and sampling strategies needed for large-scale deployment. Trade-offs between precision and scalability must be managed.

\bibliographystyle{apalike}
\bibliography{references}

\end{document}
