\documentclass[11pt]{article}
\usepackage[margin=1in]{geometry}
\usepackage{hyperref}
\usepackage{xcolor}
\usepackage{enumitem}
\usepackage{tikz}

\definecolor{urgent}{RGB}{220,20,60}
\definecolor{important}{RGB}{255,140,0}
\definecolor{normal}{RGB}{34,139,34}

\title{\textbf{Submission Checklist}\\
\large COEC Framework → Artificial Life}
\author{Pre-Submission Quality Control}
\date{\today}

\begin{document}

\maketitle

\section*{Overview}

This checklist ensures your COEC manuscript meets all requirements for submission to \textit{Artificial Life}. Complete all items before uploading to Manuscript Central.

\textbf{Target Timeline:} 6 weeks from start to submission

\section{Content Requirements}

\subsection{Introduction \& Framing (\textcolor{urgent}{CRITICAL})}

\begin{itemize}[label=$\square$]
    \item Opening paragraph references Langton's ``life-as-it-could-be'' vision
    \item Connection to ALife research explicitly stated in introduction
    \item Biological examples precede technical definitions
    \item Problem statement: why traditional models fail for biological computation
    \item Clear thesis: COEC as constraint-oriented alternative
    \item Citations to key ALife papers: Langton (1989), Kauffman, Wolfram, etc.
\end{itemize}

\subsection{Abstract Revision (\textcolor{urgent}{CRITICAL})}

\begin{itemize}[label=$\square$]
    \item Opens with biological motivation (not technical definition)
    \item Mentions ``life-as-it-could-be'' or similar ALife framing
    \item Word count: 240-260 words
    \item Structure: Problem → Solution → Contributions → Applications
    \item Emphasizes emergence and self-organization
    \item No citations in abstract
\end{itemize}

\subsection{Computational Examples (\textcolor{urgent}{CRITICAL})}

\begin{itemize}[label=$\square$]
    \item \textbf{Example 1:} Simple SS-COEC system (e.g., protein folding toy model)
    \begin{itemize}[label=$\circ$]
        \item Clear constraint definition
        \item State space visualization
        \item Entropy minimization demonstration
        \item Connection to actual biology
    \end{itemize}

    \item \textbf{Example 2:} DS-COEC system (e.g., cellular automaton with COEC interpretation)
    \begin{itemize}[label=$\circ$]
        \item Discrete dynamics formulation
        \item Emergent patterns shown
        \item Comparison to traditional CA analysis
    \end{itemize}

    \item \textbf{Example 3 (Optional):} CT-COEC or FP-COEC system
    \begin{itemize}[label=$\circ$]
        \item Demonstrates higher computational class
        \item Biological or synthetic biology relevance
    \end{itemize}
\end{itemize}

\subsection{Biological Grounding (\textcolor{important}{HIGH PRIORITY})}

\begin{itemize}[label=$\square$]
    \item Each computational class has biological example
    \item Molecular level: protein folding, molecular motors
    \item Cellular level: gene regulatory networks, signaling pathways
    \item Organismal level: neural dynamics, morphogenesis
    \item Ecological level: collective behavior, ecosystem dynamics
    \item Examples are concrete with specific systems named
\end{itemize}

\subsection{Theoretical Connections (\textcolor{normal}{IMPORTANT})}

\begin{itemize}[label=$\square$]
    \item Free Energy Principle connection explained clearly
    \item Information theory integration (Shannon entropy, mutual information)
    \item Thermodynamics (non-equilibrium systems, dissipative structures)
    \item Complexity theory (emergence, self-organization)
    \item Each connection has implications discussed
\end{itemize}

\subsection{Testable Predictions (\textcolor{important}{HIGH PRIORITY})}

\begin{itemize}[label=$\square$]
    \item Section on experimental validation
    \item Specific predictions for at least 3 biological systems
    \item Predictions are falsifiable and measurable
    \item Experimental approaches suggested
    \item Connection to current research programs
\end{itemize}

\subsection{Design Principles (\textcolor{normal}{IMPORTANT})}

\begin{itemize}[label=$\square$]
    \item Section on engineering applications
    \item Guidelines for designing COEC systems
    \item Examples in synthetic biology
    \item Examples in neuromorphic computing
    \item Examples in bio-inspired robotics
    \item Connection to ``wet'' and ``hard'' ALife
\end{itemize}

\section{Writing \& Style}

\subsection{Language \& Accessibility}

\begin{itemize}[label=$\square$]
    \item Technical terms defined on first use
    \item Mathematical notation explained intuitively
    \item Biological examples before mathematical formalism
    \item Jargon minimized; concepts explained clearly
    \item Active voice used where appropriate
    \item Transitions between sections smooth
\end{itemize}

\subsection{ALife Vocabulary Integration}

\begin{itemize}[label=$\square$]
    \item Uses ``emergence'' and ``self-organization'' appropriately
    \item References ``substrate-independence'' explicitly
    \item Mentions ``life-as-it-could-be'' paradigm
    \item Discusses ``soft''/``wet''/``hard'' ALife domains
    \item Connects to ``bottom-up'' vs. ``top-down'' approaches
\end{itemize}

\section{Technical Requirements}

\subsection{Manuscript Format}

\begin{itemize}[label=$\square$]
    \item Uses ALife Overleaf template or equivalent LaTeX format
    \item Title: Clear and descriptive (60-100 characters)
    \item Running head: Short title (max 40 characters)
    \item Author information complete with ORCID
    \item All sections properly numbered
    \item Headers and footers correct
\end{itemize}

\subsection{Length \& Structure}

\begin{itemize}[label=$\square$]
    \item Total word count: 12,000-18,000 (Review Article)
    \item Abstract: 240-260 words
    \item Introduction: 1,500-2,000 words
    \item Main content organized into clear sections
    \item Conclusion: 500-800 words with clear take-aways
\end{itemize}

\subsection{Figures \& Tables}

\begin{itemize}[label=$\square$]
    \item All figures referenced in text
    \item Figure captions descriptive and standalone
    \item Figures high resolution (at least 300 DPI)
    \item Color figures use colorblind-friendly palettes
    \item Figure 1: Conceptual overview of COEC framework
    \item Figure 2: Nine-class taxonomy visualization
    \item Figures 3-5: Computational example demonstrations
    \item Figure 6: Cross-scale applications diagram
\end{itemize}

\subsection{References \& Citations}

\begin{itemize}[label=$\square$]
    \item APA citation style used throughout
    \item Bibliography: 60-100 references
    \item \textbf{Key ALife citations included:}
    \begin{itemize}[label=$\circ$]
        \item Langton (1989) - Artificial Life definition
        \item Kauffman (1993) - Origins of Order
        \item Wolfram (2002) - A New Kind of Science
        \item Holland (1992) - Emergence
        \item Ray (1991) - Tierra
        \item Adami et al. (2000) - Evolution of biological complexity
    \end{itemize}
    \item Recent ALife papers cited (2015-2024)
    \item Free Energy Principle: Friston et al.
    \item Information theory: Shannon, Jaynes
    \item All URLs working and accessible
\end{itemize}

\subsection{Mathematical Content}

\begin{itemize}[label=$\square$]
    \item All equations numbered and referenced
    \item Notation consistent throughout
    \item Variables defined in text or notation table
    \item Complex equations broken into components
    \item Intuitive explanations accompany formal definitions
\end{itemize}

\section{Supplementary Materials}

\subsection{Code \& Data}

\begin{itemize}[label=$\square$]
    \item Computational examples have code available
    \item Code deposited in repository (GitHub, Zenodo)
    \item Repository link included in manuscript
    \item Code documented with README
    \item License specified (prefer MIT or GPL)
\end{itemize}

\subsection{Additional Documentation}

\begin{itemize}[label=$\square$]
    \item Supplementary mathematical derivations (if needed)
    \item Extended biological examples
    \item Additional figures/visualizations
    \item Video demonstrations of computational examples (optional)
\end{itemize}

\section{Pre-Submission Review}

\subsection{Internal Review}

\begin{itemize}[label=$\square$]
    \item Manuscript reviewed by colleague/advisor
    \item Feedback addressed comprehensively
    \item Fresh-eyes read after 2-3 day break
    \item All co-authors approved final version
\end{itemize}

\subsection{Proofreading}

\begin{itemize}[label=$\square$]
    \item Spell check completed
    \item Grammar checked (Grammarly or similar)
    \item Mathematical notation verified
    \item All references formatted correctly
    \item Figure numbering consistent
    \item No placeholder text (e.g., ``TODO'', ``XXX'')
\end{itemize}

\subsection{Quality Checks}

\begin{itemize}[label=$\square$]
    \item Abstract compelling and accurate
    \item Introduction hooks reader immediately
    \item Figures tell story independently
    \item Conclusion provides clear take-home messages
    \item Paper flows logically from start to finish
    \item Technical rigor balanced with accessibility
\end{itemize}

\section{Submission Process}

\subsection{Before Upload}

\begin{itemize}[label=$\square$]
    \item Manuscript Central account created
    \item ORCID iD obtained and linked
    \item Suggested reviewers list prepared (4-6 names)
    \item Excluded reviewers list prepared (if any)
    \item Keywords selected (5-7 relevant terms)
    \item Cover letter finalized
\end{itemize}

\subsection{Required Files}

\begin{itemize}[label=$\square$]
    \item Manuscript PDF (generated from LaTeX)
    \item Source files (LaTeX .tex files)
    \item All figures (individual high-res files)
    \item Cover letter PDF
    \item Supplementary materials (if applicable)
\end{itemize}

\subsection{Submission Metadata}

\begin{itemize}[label=$\square$]
    \item Article type: Review Article
    \item Title matches manuscript
    \item Abstract pasted into submission form
    \item Keywords entered: \textit{(suggested)}
    \begin{itemize}[label=$\circ$]
        \item emergent computation
        \item constraint-oriented systems
        \item biological computation
        \item substrate-independence
        \item self-organization
        \item Free Energy Principle
        \item life-as-it-could-be
    \end{itemize}
    \item Funding information (if applicable)
    \item Conflicts of interest statement
\end{itemize}

\subsection{Final Checks}

\begin{itemize}[label=$\square$]
    \item All files uploaded correctly
    \item PDF renders properly in viewer
    \item Figures display correctly
    \item Metadata complete and accurate
    \item Cover letter persuasive and professional
    \item Optional: Note about 30th anniversary special issue
\end{itemize}

\section{Post-Submission}

\subsection{Confirmation}

\begin{itemize}[label=$\square$]
    \item Submission confirmation email received
    \item Manuscript ID number recorded
    \item Submission date documented
    \item Expected review timeline noted (3-5 months)
\end{itemize}

\subsection{Next Steps}

\begin{itemize}[label=$\square$]
    \item Begin work on next paper (Biocomputing → BioSystems)
    \item Prepare presentation slides summarizing COEC
    \item Draft blog post or accessible summary
    \item Identify relevant conferences for presentation
    \item Set calendar reminder to check submission status
\end{itemize}

\section*{Final Reminders}

\textbf{Critical Success Factors:}
\begin{enumerate}
    \item \textcolor{urgent}{ALife framing} — Must connect explicitly to field
    \item \textcolor{urgent}{Concrete examples} — 2-3 computational demonstrations required
    \item \textcolor{urgent}{Biological grounding} — Not just abstract math
\end{enumerate}

\textbf{Common Pitfalls to Avoid:}
\begin{enumerate}
    \item Too abstract without concrete systems
    \item Missing ALife context and citations
    \item Unfalsifiable claims without testable predictions
    \item Poor writing or unclear exposition
    \item Wrong length (under 12k or over 20k words)
\end{enumerate}

\vspace{1em}
\textbf{You're ready to submit when:}
\begin{itemize}
    \item All \textcolor{urgent}{CRITICAL} items completed
    \item 90\%+ of checklist items checked
    \item Manuscript reviewed by at least one other person
    \item You feel confident (but okay to still be nervous!)
\end{itemize}

\vspace{2em}
\hrule
\vspace{0.5em}
\textit{Good luck! Remember: Rejection is normal (75\% rejection rate). If rejected, revise and submit to backup venues. Keep going until published!}

\end{document}
