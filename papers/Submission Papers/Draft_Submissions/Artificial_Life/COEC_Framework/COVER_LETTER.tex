\documentclass[11pt]{letter}
\usepackage[margin=1in]{geometry}
\usepackage{hyperref}

\signature{Rohan Vinaik}
\address{%
[Your Institution/Affiliation]\\
[Your Address]\\
[Your Email]\\
[Your Phone]
}

\begin{document}

\begin{letter}{%
Editor-in-Chief\\
\textit{Artificial Life}\\
MIT Press Journals\\
One Rogers Street\\
Cambridge, MA 02142-1209
}

\opening{Dear Editor,}

I am pleased to submit our manuscript titled ``\textbf{Constraint-Oriented Emergent Computation: A Formal Framework for Biological and Artificial Systems}'' for consideration as a Review Article in \textit{Artificial Life}.

This manuscript presents a novel theoretical framework that unifies the understanding of computation across biological and artificial systems. By formalizing computation as the trajectory of physical systems through constrained state spaces, the COEC framework bridges scales from molecular interactions to ecosystem dynamics, addressing core themes in artificial life research including emergence, self-organization, and the synthesis of life-like phenomena.

\subsection*{Key Contributions}

\begin{itemize}
    \item \textbf{Substrate-independent mathematical framework:} COEC provides a formal language for understanding emergent computation in any physical medium, directly advancing ALife's ``life-as-it-could-be'' vision.

    \item \textbf{Taxonomy of computational classes:} We introduce nine distinct classes (SS-COEC through Sheaf-COEC) spanning Sub-Turing to Hyper-Turing capabilities, each characterized by unique constraint dynamics.

    \item \textbf{Theoretical integration:} The framework connects information theory, thermodynamics, and the Free Energy Principle, providing unified principles for understanding biological computation.

    \item \textbf{Testable predictions:} COEC generates specific predictions for experimental validation across multiple scales, from molecular self-assembly to collective behavior.

    \item \textbf{Design principles:} The framework offers concrete guidelines for engineering computational systems across ``soft,'' ``wet,'' and ``hard'' ALife domains.
\end{itemize}

\subsection*{Relevance to Artificial Life}

This work aligns closely with \textit{Artificial Life}'s mission to extend ``the horizons of biological research beyond life-as-we-know-it and into the domain of life-as-it-could-be'' (Langton, 1989). COEC builds on foundational ALife concepts—self-organization (Kauffman), substrate-independence (Langton), and emergent complexity (Wolfram)—while providing new mathematical tools for both analyzing natural systems and synthesizing novel computational substrates.

The framework addresses a longstanding gap in ALife research: how to formalize computation as it \textit{actually occurs} in living systems—not through symbolic algorithms or centralized control, but through the physical dynamics of constrained matter. This perspective enables:

\begin{itemize}
    \item Analysis of biological computation across all scales
    \item Design of novel bio-inspired computational architectures
    \item Understanding of how life-like properties emerge from physical constraints
    \item Synthesis of new forms of computation beyond traditional paradigms
\end{itemize}

\subsection*{30th Anniversary Special Issue}

We note this manuscript as a potential contribution to the 30th anniversary special issue (Volume 30, Issue 1), given its theoretical grounding in Langton's foundational vision and its synthesis of three decades of ALife research. However, we are happy for it to be considered for any appropriate issue at the editor's discretion.

\subsection*{Manuscript Details}

\begin{itemize}
    \item \textbf{Type:} Review Article
    \item \textbf{Word Count:} Approximately 15,000 words
    \item \textbf{Figures:} 6-8 (including computational examples and taxonomy visualization)
    \item \textbf{Supplementary Materials:} Code examples and computational demonstrations (available upon request)
\end{itemize}

\subsection*{Author Information}

\textbf{Rohan Vinaik} is an independent researcher specializing in theoretical frameworks for biological computation, hyperdimensional computing, and computational ontology. This work synthesizes insights from computational biology, theoretical computer science, and artificial life research.

\subsection*{Competing Interests}

There are no competing interests to declare. This work has not been submitted elsewhere and will not be submitted to other journals while under consideration at \textit{Artificial Life}.

\subsection*{Suggested Reviewers}

We respectfully suggest the following potential reviewers with expertise in theoretical artificial life, biological computation, and emergent systems:

\begin{enumerate}
    \item \textbf{Dr. [Name]} — [Institution] (expertise in self-organization and emergence)
    \item \textbf{Dr. [Name]} — [Institution] (expertise in artificial chemistries and molecular computation)
    \item \textbf{Dr. [Name]} — [Institution] (expertise in Free Energy Principle and active inference)
    \item \textbf{Dr. [Name]} — [Institution] (expertise in computational theory and complexity)
\end{enumerate}

Thank you for considering this manuscript for publication in \textit{Artificial Life}. We believe COEC offers valuable new conceptual tools for the artificial life community and look forward to your assessment.

\closing{Sincerely,}

\end{letter}

\end{document}
