\documentclass[11pt]{letter}
\usepackage[margin=1in]{geometry}
\usepackage{hyperref}

\signature{Rohan Vinaik\\Independent Researcher}
\address{Rohan Vinaik\\rohan.vinaik@gmail.com}

\begin{document}

\begin{letter}{Editor-in-Chief\\BioSystems\\Elsevier}

\opening{Dear Editor,}

I submit for consideration in \textit{BioSystems} the manuscript ``Hyperdimensional Computing in Biological Substrates: Framework and Implementation Principles.''

This work presents a novel theoretical framework demonstrating how biological systems—from proteins to neural networks—naturally implement hyperdimensional computing (HDC) principles through their physical properties. By formalizing the mapping between biological encoding mechanisms and HDC operations, this framework provides both analytical tools for understanding biological information processing and design principles for synthetic biological computing systems.

\textbf{Key contributions include:}

\begin{itemize}
    \item Mathematical formalization of biological substrates (proteins, DNA/RNA, neural tissue) as HDC implementers, with quantitative dimensionality analysis ranging from $10^2$ (protein conformations) to $10^9$ (neural populations) dimensions

    \item Analysis of how synthetic amino acid incorporation expands protein encoding capacity by 14-30\%, providing explicit engineering handles for protein-based HDC systems

    \item Demonstration that fundamental biological mechanisms—allosteric regulation, strand displacement, synaptic integration—naturally implement core HDC operations of binding, bundling, and permutation

    \item Design principles for engineering synthetic biological computing systems that exploit HDC's intrinsic noise tolerance, distributed representation, and similarity-based operations

    \item Applications in synthetic biology, including distributed biosensing, cellular programming (with stem cell differentiation case study achieving 95\% efficiency), and adaptive cellular behaviors through epigenetic memory
\end{itemize}

This work addresses \textit{BioSystems}' core themes of self-organizing information systems and theoretical biology, offering a unifying computational perspective on biological information processing across scales. Unlike traditional approaches that impose digital logic frameworks onto biological substrates, this framework reveals computational paradigms that biology naturally excels at—those leveraging high-dimensional continuous state spaces, noise tolerance through distributed representation, and similarity-based pattern matching.

The framework synthesizes insights from molecular biology, synthetic biology, computer science, and applied mathematics. It provides both conceptual tools for understanding natural biological computation (why is biological computation so noisy? why are regulatory networks so complex?) and practical guidance for engineering synthetic biocomputing systems. The synthetic biology applications demonstrate the framework's utility beyond pure theory.

I believe this work represents a significant contribution to theoretical biology and synthetic biology, and would be of strong interest to \textit{BioSystems} readers. The manuscript is approximately 9,000 words with 1 table and 12 references (expandable to 40-60 with additional biological and HDC literature as needed).

This manuscript has not been published elsewhere and is not under consideration by any other journal. All authors have approved the manuscript for submission.

Thank you for your consideration. I look forward to your response.

\closing{Sincerely,}

\end{letter}

\end{document}
